\documentclass[12pt]{article}
\usepackage[utf8]{inputenc}
\usepackage{amsmath}
\usepackage{amssymb}
\usepackage[margin=2cm]{geometry}

\title{Formalización Matemática del Sistema Difuso de Sedentarismo}
\author{Luis Ángel Martínez - UACH}
\date{\today}

\begin{document}
\maketitle

\section{Notación}

Sea la unidad de análisis la \textbf{semana} $i \in \{1, \dots, n\}$. Para cada semana construimos el vector de rasgos normalizados a $[0,1]$:

\[
\mathbf{x}_i = \begin{bmatrix}
\text{Act}_{p50} \\
\text{Sup}_{p50} \\
\text{HRV}_{p50} \\
\Delta\text{Card}_{p50}
\end{bmatrix}_i \in [0,1]^4
\]

\section{Funciones de Membresía}

Para cada variable $x_{ij}$ definimos tres etiquetas lingüísticas: \{Baja, Media, Alta\} con funciones de membresía \textbf{triangulares}:

\[
\mu^{\text{Baja}}(x; a, b, c), \quad \mu^{\text{Media}}(x; a, b, c), \quad \mu^{\text{Alta}}(x; a, b, c)
\]

donde la triangular estándar es:

\[
\mu(x; a, b, c) = \begin{cases}
0, & x \le a \text{ o } x \ge c \\
\frac{x - a}{b - a}, & a < x < b \\
\frac{c - x}{c - b}, & b \le x < c
\end{cases}
\]

Los parámetros $(a, b, c)$ se determinan por percentiles:
\begin{itemize}
    \item Baja: $(p_{10}, p_{25}, p_{40})$
    \item Media: $(p_{35}, p_{50}, p_{65})$
    \item Alta: $(p_{60}, p_{80}, p_{90})$
\end{itemize}

\section{Vector de Membresías}

Para cada semana $i$, el vector de membresías es:

\[
\boldsymbol{\mu}_i = \begin{bmatrix}
\mu_{\text{Act}_B}, \mu_{\text{Act}_M}, \mu_{\text{Act}_A}, \\
\mu_{\text{Sup}_B}, \mu_{\text{Sup}_M}, \mu_{\text{Sup}_A}, \\
\mu_{\text{HRV}_B}, \mu_{\text{HRV}_M}, \mu_{\text{HRV}_A}, \\
\mu_{\text{DC}_B}, \mu_{\text{DC}_M}, \mu_{\text{DC}_A}
\end{bmatrix}_i \in [0,1]^{12}
\]

\section{Base de Reglas}

Definimos $R = 5$ reglas (conjunciones de antecedentes $\to$ consecuente):

\subsection*{Matriz de Antecedentes $\mathbf{B} \in \{0,1\}^{5 \times 12}$}

\[
\mathbf{B} = \begin{bmatrix}
1 & 0 & 0 & 1 & 0 & 0 & 0 & 0 & 0 & 0 & 0 & 0 \\ % R1: Act_B, Sup_B
0 & 0 & 1 & 0 & 0 & 1 & 0 & 0 & 0 & 0 & 0 & 0 \\ % R2: Act_A, Sup_A
0 & 0 & 0 & 0 & 0 & 0 & 1 & 0 & 0 & 1 & 0 & 0 \\ % R3: HRV_B, DC_B
0 & 1 & 0 & 0 & 0 & 0 & 0 & 1 & 0 & 0 & 0 & 0 \\ % R4: Act_M, HRV_M
1 & 0 & 0 & 0 & 1 & 0 & 0 & 0 & 0 & 0 & 0 & 0   % R5: Act_B, Sup_M
\end{bmatrix}
\]

Columnas: $[\text{Act}_B, \text{Act}_M, \text{Act}_A, \text{Sup}_B, \text{Sup}_M, \text{Sup}_A, \text{HRV}_B, \text{HRV}_M, \text{HRV}_A, \text{DC}_B, \text{DC}_M, \text{DC}_A]$

\subsection*{Matriz de Consecuentes $\mathbf{C}_{\text{out}} \in \mathbb{R}^{5 \times 3}$}

\[
\mathbf{C}_{\text{out}} = \begin{bmatrix}
0 & 0 & 1.0 \\ % R1 -> Alto
1.0 & 0 & 0 \\ % R2 -> Bajo
0 & 0 & 1.0 \\ % R3 -> Alto
0 & 1.0 & 0 \\ % R4 -> Medio
0 & 0 & 0.7   % R5 -> Alto (peso 0.7)
\end{bmatrix}
\]

Columnas: $[\text{Sed}_{\text{Bajo}}, \text{Sed}_{\text{Medio}}, \text{Sed}_{\text{Alto}}]$

\section{Inferencia Mamdani}

\subsection{Activación (AND = mín)}

Para cada regla $r$, calculamos la activación:

\[
w_{i,r} = \min \{ \mu_{i,j} \mid B_{rj} = 1 \}
\]

\subsection{Agregación (suma ponderada)}

\[
\mathbf{s}_i = \mathbf{w}_i^\top \mathbf{C}_{\text{out}} = \begin{bmatrix}
s_{i,\text{Bajo}} \\
s_{i,\text{Medio}} \\
s_{i,\text{Alto}}
\end{bmatrix}
\]

\subsection{Defuzzificación (centroide discreto)}

\[
\text{score}_i = \frac{0.2 \cdot s_{i,\text{Bajo}} + 0.5 \cdot s_{i,\text{Medio}} + 0.8 \cdot s_{i,\text{Alto}}}{s_{i,\text{Bajo}} + s_{i,\text{Medio}} + s_{i,\text{Alto}}}
\]

\subsection{Binarización}

\[
\hat{y}_i = \mathbb{1}[\text{score}_i \ge \tau], \quad \tau = 0.30
\]

\section{Pseudocódigo}

\begin{verbatim}
Input: X ∈ ℝ^{n×4} (weekly features, normalized)
Output: scores ∈ [0,1]^n, labels ∈ {0,1}^n

1. Para cada semana i:
   1.1. Fuzzificar: μ_i ← fuzzify(x_i; MF_params)
   1.2. Activar reglas: w_i[r] ← min(μ_i[j] : B[r,j]=1), r=1..5
   1.3. Agregar: s_i ← w_i^T · C_out
   1.4. Defuzzificar: score_i ← (0.2·s_B + 0.5·s_M + 0.8·s_A) / ||s_i||_1
   1.5. Binarizar: ŷ_i ← [score_i ≥ τ]
2. Retornar: scores, labels
\end{verbatim}

\end{document}
