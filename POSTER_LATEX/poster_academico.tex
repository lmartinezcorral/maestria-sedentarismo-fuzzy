%!TEX program = pdflatex
% ============================================================================
% POSTER ACADÉMICO: Sistema de Inferencia Difusa para Sedentarismo
% Template usando tikzposter
% ============================================================================

\documentclass[25pt, a0paper, portrait]{tikzposter}

% ============================================================================
% PAQUETES
% ============================================================================
\usepackage[utf8]{inputenc}
\usepackage[spanish]{babel}
\usepackage{graphicx}
\usepackage{booktabs}
\usepackage{amsmath}
\usepackage{multirow}
\usepackage{tabularx}

% ============================================================================
% CONFIGURACIÓN DEL POSTER
% ============================================================================

% Tema del poster (opciones: Default, Rays, Basic, Simple, Envelope, Wave, Board, Autumn, Desert)
\usetheme{Autumn}

% Colores (opciones: Default, Australia, Britain, Sweden, Spain, Russia, Denmark, Germany)
\usecolorstyle[colorPalette=BrownBlueOrange]{Germany}

% Título y autores
\title{Sistema de Inferencia Difusa para Evaluación del Comportamiento Sedentario mediante Datos de Wearables}

\author{Luis Ángel Martínez}

\institute{Maestría en Ciencias, Semestre 3 \\ Universidad [Nombre] \\ luis.martinez@institution.edu}

\titlegraphic{\includegraphics[width=0.08\textwidth]{logo_universidad.png}}

% ============================================================================
% INICIO DEL DOCUMENTO
% ============================================================================

\begin{document}

\maketitle

% ============================================================================
% COLUMNA IZQUIERDA
% ============================================================================

\begin{columns}

\column{0.33} % Primera columna (33%)

% ----------------------------------------------------------------------------
% BLOQUE 1: INTRODUCCIÓN Y OBJETIVOS
% ----------------------------------------------------------------------------

\block{1. Introducción y Motivación}{
\textbf{Contexto:} El sedentarismo es un factor de riesgo cardiovascular crítico. Los datos de wearables (Apple Watch) permiten evaluación objetiva en vida libre.

\vspace{0.5cm}

\textbf{Desafío:} 400 kcal gastadas $\neq$ impacto fisiológico equivalente entre usuarios con distinto metabolismo basal (TMB).

\vspace{0.5cm}

\textbf{Solución:} Variables derivadas normalizadas por antropometría y exposición.

\vspace{1cm}

\textbf{OBJETIVO PRINCIPAL:}

\begin{itemize}
    \item Desarrollar y validar un sistema de inferencia difusa tipo Mamdani para clasificar el sedentarismo semanal
    \item Contrastar con verdad operativa de clustering no supervisado (K-means)
    \item Alta sensibilidad (screening poblacional) + reglas interpretables
\end{itemize}
}

% ----------------------------------------------------------------------------
% BLOQUE 2: DATOS Y COHORTE
% ----------------------------------------------------------------------------

\block{2. Datos y Cohorte}{

\textbf{Cohorte:} 10 adultos (5M/5H), seguimiento multianual

\textbf{Unidad de análisis:} 1,337 semanas válidas (agregación diaria $\rightarrow$ semanal)

\vspace{0.5cm}

\begin{center}
\begin{tabular}{lccccc}
\toprule
\textbf{Usuario} & \textbf{Sexo} & \textbf{Edad} & \textbf{Peso} & \textbf{TMB} & \textbf{Semanas} \\
 &  & (años) & (kg) & (kcal/d) & válidas \\
\midrule
u1 (ale) & M & 34 & 68 & 1411 & 149 \\
u6 (fidel) & H & 34 & 100 & 1958 & 278 \\
u9 (lmartinez) & H & 32 & 124 & 2241 & 298 \\
u10 (vane) & M & 28 & 58 & 1304 & 131 \\
\bottomrule
\end{tabular}
\end{center}

\vspace{0.5cm}

\textbf{Rango de TMB:} 1,304--2,241 kcal/día (variabilidad 72\%)

\textbf{Variables base:} Actividad física, FC (reposo/caminata), HRV (SDNN), gasto calórico, horas monitoreadas
}

% ----------------------------------------------------------------------------
% BLOQUE 3: VARIABLES DERIVADAS CLAVE
% ----------------------------------------------------------------------------

\block{3. Variables Derivadas Clave}{

\textbf{1. Actividad\_relativa} (normalizada por exposición):
\begin{equation*}
\text{Actividad\_relativa} = \frac{\text{min\_movimiento}}{60 \times \text{hrs\_monitoreadas}}
\end{equation*}

\textit{Rationale:} Corrige por tiempo de uso del reloj.

\vspace{0.5cm}

\textbf{2. Superávit\_calórico\_basal} (ajustado por TMB):
\begin{equation*}
\text{Superávit} = \frac{\text{Gasto\_activo} \times 100}{\text{TMB}}
\end{equation*}

Donde TMB (Mifflin-St Jeor):
\begin{align*}
\text{Hombres:} & \quad \text{TMB} = 10 \cdot \text{peso} + 6.25 \cdot \text{altura} - 5 \cdot \text{edad} + 5 \\
\text{Mujeres:} & \quad \text{TMB} = 10 \cdot \text{peso} + 6.25 \cdot \text{altura} - 5 \cdot \text{edad} - 161
\end{align*}

\textit{Rationale:} Ajusta por metabolismo basal individual.

\vspace{0.5cm}

\textbf{3. HRV\_SDNN} (variabilidad cardíaca):
\begin{itemize}
    \item Alta HRV ($>60$ ms): Tono vagal saludable
    \item Baja HRV ($<40$ ms): Desacondicionamiento, estrés
\end{itemize}

\vspace{0.5cm}

\textbf{4. Delta\_cardiaco} (respuesta al esfuerzo):
\begin{equation*}
\Delta_{\text{cardiaco}} = \text{FC}_{\text{caminata}} - \text{FC}_{\text{reposo}}
\end{equation*}
}

% ============================================================================
% COLUMNA CENTRAL
% ============================================================================

\column{0.34} % Segunda columna (34%)

% ----------------------------------------------------------------------------
% BLOQUE 4: PIPELINE METODOLÓGICO
% ----------------------------------------------------------------------------

\block{4. Pipeline Metodológico (5 Fases)}{

\begin{center}
\fbox{\parbox{0.9\linewidth}{
\textbf{FASE 1:} Preprocesamiento + Imputación Jerárquica\\
\textit{Gates: Hard no-wear, Soft low-activity, Normal}\\
\textit{Rolling mediana (solo pasado, sin leak temporal)}\\[0.3cm]
$\downarrow$\\[0.3cm]
\textbf{FASE 2:} Variables Derivadas\\
\textit{Actividad\_relativa, Superávit\_calórico (reemplazan originales)}\\[0.3cm]
$\downarrow$\\[0.3cm]
\textbf{FASE 3:} Agregación Semanal\\
\textit{8 features: p50 (mediana) + IQR por variable}\\[0.3cm]
$\downarrow$\\[0.3cm]
\textbf{FASE 4:} Clustering K-Means (Verdad Operativa)\\
\textit{K=2 óptimo (Silhouette=0.232)}\\[0.3cm]
$\downarrow$\\[0.3cm]
\textbf{FASE 5:} Sistema Difuso + Validación\\
\textit{4 entradas $\times$ 3 etiquetas, 5 reglas, $\tau=0.30$}
}}
\end{center}
}

% ----------------------------------------------------------------------------
% BLOQUE 5: CLUSTERING K=2
% ----------------------------------------------------------------------------

\block{5. Clustering K-Means (Verdad Operativa)}{

\textbf{K-sweep (K=2..6):} K=2 óptimo

\vspace{0.5cm}

\begin{center}
\begin{tabular}{cccl}
\toprule
\textbf{K} & \textbf{Silhouette} & \textbf{Davies-B} & \textbf{Tamaños} \\
\midrule
\textbf{2} & \textbf{0.232} & 2.058 & \{0: 402, 1: 935\} \\
3 & 0.195 & 1.721 & \{0: 685, 1: 235, 2: 417\} \\
4 & 0.192 & 1.422 & 4 clusters \\
\bottomrule
\end{tabular}
\end{center}

\vspace{0.5cm}

\textbf{Perfiles de K=2:}

\begin{itemize}
    \item \textbf{Cluster 0 (Bajo Sedentarismo):} 402 semanas (30\%)
    \begin{itemize}
        \item Actividad\_rel = 0.160, Superávit = 45.4\%
    \end{itemize}
    \item \textbf{Cluster 1 (Alto Sedentarismo):} 935 semanas (70\%)
    \begin{itemize}
        \item Actividad\_rel = 0.116, Superávit = 25.4\%
    \end{itemize}
\end{itemize}
}

% ----------------------------------------------------------------------------
% BLOQUE 6: FUNCIONES DE MEMBRESÍA
% ----------------------------------------------------------------------------

\block{6. Funciones de Membresía (Triangulares por Percentiles)}{

\begin{center}
\includegraphics[width=0.48\linewidth]{../analisis_u/fuzzy/plots/MF_Actividad_relativa_p50.png}
\includegraphics[width=0.48\linewidth]{../analisis_u/fuzzy/plots/MF_Superavit_calorico_basal_p50.png}

\vspace{0.3cm}

\includegraphics[width=0.48\linewidth]{../analisis_u/fuzzy/plots/MF_HRV_SDNN_p50.png}
\includegraphics[width=0.48\linewidth]{../analisis_u/fuzzy/plots/MF_Delta_cardiaco_p50.png}
\end{center}

\vspace{0.3cm}

\textbf{Diseño data-driven:} Percentiles p10-p25-p40, p35-p50-p65, p60-p75-p90
}

% ----------------------------------------------------------------------------
% BLOQUE 7: SISTEMA DIFUSO
% ----------------------------------------------------------------------------

\block{7. Sistema de Inferencia Difusa (5 Reglas Mamdani)}{

\textbf{Entradas (4):} Actividad\_relativa, Superávit\_calórico, HRV\_SDNN, Delta\_cardiaco

\vspace{0.3cm}

\textbf{Reglas (ejemplos):}

\begin{enumerate}
    \item SI Actividad es \textit{Baja} Y Superávit es \textit{Bajo} $\Rightarrow$ Sedentarismo \textit{Alto}
    \item SI Actividad es \textit{Alta} Y Superávit es \textit{Alto} $\Rightarrow$ Sedentarismo \textit{Bajo}
    \item SI HRV es \textit{Baja} Y Delta es \textit{Alta} $\Rightarrow$ Sedentarismo \textit{Alto}
    \item SI Actividad es \textit{Media} Y HRV es \textit{Media} $\Rightarrow$ Sedentarismo \textit{Medio}
    \item SI Actividad es \textit{Baja} Y Superávit es \textit{Medio} $\Rightarrow$ Sedentarismo \textit{Medio-Alto}
\end{enumerate}

\vspace{0.3cm}

\textbf{Defuzzificación:} Centroide $\rightarrow$ Score $\in [0, 1]$

\textbf{Binarización:} $\tau = 0.30$ (maximiza F1 vs. clusters)
}

% ============================================================================
% COLUMNA DERECHA
% ============================================================================

\column{0.33} % Tercera columna (33%)

% ----------------------------------------------------------------------------
% BLOQUE 8: RESULTADOS - MÉTRICAS
% ----------------------------------------------------------------------------

\block{8. Resultados: Métricas de Validación}{

\begin{center}
\Large
\textbf{F1-Score: 0.840}\\[0.3cm]
\textbf{Recall: 97.6\%}\\[0.3cm]
\textbf{Accuracy: 74.0\%}\\[0.3cm]
\normalsize
Precision: 73.7\% \quad MCC: 0.294 \quad $\tau = 0.30$
\end{center}

\vspace{0.5cm}

\textbf{Matriz de Confusión:}

\begin{center}
\begin{tabular}{l|cc|c}
\toprule
 & \textbf{Fuzzy: Bajo} & \textbf{Fuzzy: Alto} & \textbf{Total} \\
\midrule
\textbf{Cluster: Bajo} & TN = 77 & FP = 325 & 402 \\
\textbf{Cluster: Alto} & FN = 22 & TP = 913 & 935 \\
\midrule
\textbf{Total} & 99 & 1,238 & 1,337 \\
\bottomrule
\end{tabular}
\end{center}

\vspace{0.5cm}

\begin{center}
\includegraphics[width=0.9\linewidth]{../analisis_u/fuzzy/plots/confusion_matrix.png}
\end{center}
}

% ----------------------------------------------------------------------------
% BLOQUE 9: CURVAS Y DISTRIBUCIÓN
% ----------------------------------------------------------------------------

\block{9. Curva Precision-Recall y Distribución de Scores}{

\begin{center}
\includegraphics[width=0.48\linewidth]{../analisis_u/fuzzy/plots/pr_curve.png}
\includegraphics[width=0.48\linewidth]{../analisis_u/fuzzy/plots/score_distribution_by_cluster.png}
\end{center}

\vspace{0.3cm}

\textbf{Punto óptimo:} $\tau=0.30$ maximiza F1 (balance Precision-Recall)
}

% ----------------------------------------------------------------------------
% BLOQUE 10: CONCORDANCIA POR USUARIO
% ----------------------------------------------------------------------------

\block{10. Concordancia por Usuario (Heterogeneidad)}{

\textbf{Concordancia media:} 70.0\% (rango: 27.7\%--99.3\%)

\vspace{0.5cm}

\begin{center}
\begin{tabular}{lccc}
\toprule
\textbf{Usuario} & \textbf{Concordancia} & \textbf{F1} & \textbf{Recall} \\
\midrule
u1 (ale) & \textbf{99.3\%} & 0.997 & 1.000 \\
u7 (kevin) & \textbf{94.7\%} & 0.973 & 1.000 \\
u6 (fidel) & 81.7\% & 0.898 & 0.982 \\
u10 (vane) & 80.9\% & 0.895 & 1.000 \\
u3 (christina) & 27.7\% & 0.215 & 0.875 \\
u8 (legarda) & 44.0\% & 0.462 & 0.868 \\
\bottomrule
\end{tabular}
\end{center}

\vspace{0.3cm}

\textbf{Alta concordancia ($>90\%$):} Patrones estables (u1, u7)

\textbf{Baja concordancia ($<50\%$):} Alta variabilidad intra-semanal (u3, u8)
}

% ----------------------------------------------------------------------------
% BLOQUE 11: CONCLUSIONES
% ----------------------------------------------------------------------------

\block{11. Conclusiones y Próximos Pasos}{

\textbf{CONCLUSIONES:}

\begin{enumerate}
    \item \textbf{Sistema validado:} Convergencia robusta con clustering (F1=0.84, Recall=97.6\%)
    \item \textbf{Alta sensibilidad:} Minimiza falsos negativos $\rightarrow$ Screening poblacional
    \item \textbf{Variables relevantes:} Actividad\_relativa y Superávit\_calórico (principales discriminadores)
    \item \textbf{Interpretabilidad:} Reglas auditables por clínicos vs. caja negra
    \item \textbf{Trazabilidad:} Pipeline documentado, auditorías, reproducibilidad
\end{enumerate}

\vspace{0.5cm}

\textbf{PRÓXIMOS PASOS:}

\begin{itemize}
    \item \textbf{Corto plazo:} Personalización de $\tau$ por usuario, reglas moduladas por IQR
    \item \textbf{Mediano plazo:} Validación externa (cohorte $\geq 20$ usuarios), integración de sueño/estrés
    \item \textbf{Largo plazo:} Dashboard clínico, publicación científica, despliegue operativo
\end{itemize}
}

% ----------------------------------------------------------------------------
% BLOQUE 12: REFERENCIAS Y CONTACTO
% ----------------------------------------------------------------------------

\block{12. Referencias y Contacto}{

\small

\textbf{Referencias:}

[1] Troiano et al. (2008). Physical activity measured by accelerometer. \textit{Med Sci Sports Exerc}, 40(1):181-188.

[2] Thayer et al. (2010). Heart rate variability and neuroimaging. \textit{Neurosci Biobehav Rev}, 36(2):747-756.

[3] Mifflin et al. (1990). Predictive equation for resting energy expenditure. \textit{Am J Clin Nutr}, 51(2):241-247.

[4] Mamdani \& Assilian (1975). Linguistic synthesis with fuzzy logic. \textit{Int J Man Mach Stud}, 7(1):1-13.

\vspace{0.5cm}

\textbf{Contacto:}

Luis Ángel Martínez \\ Maestría en Ciencias, Semestre 3 \\ \texttt{luis.martinez@institution.edu}

\vspace{0.3cm}

\textbf{Repositorio:} \texttt{github.com/usuario/sistema-difuso-sedentarismo}

\normalsize
}

\end{columns}

\end{document}





