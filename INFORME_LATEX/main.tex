%!TEX program = pdflatex
% ============================================================================
% SISTEMA DE INFERENCIA DIFUSA PARA EVALUACIÓN DE SEDENTARISMO
% Tesis de Maestría en Ciencias - Luis Ángel Martínez
% ============================================================================

\documentclass[12pt,a4paper,twoside]{article}

% ============================================================================
% PAQUETES ESENCIALES
% ============================================================================
\usepackage[utf8]{inputenc}
\usepackage[spanish,es-tabla]{babel}
\usepackage[T1]{fontenc}
\usepackage{lmodern}

% Matemáticas
\usepackage{amsmath,amssymb,amsthm}
\usepackage{mathtools}

% Gráficos y tablas
\usepackage{graphicx}
\usepackage{booktabs}
\usepackage{multirow}
\usepackage{array}
\usepackage{longtable}
\usepackage{tabularx}
\usepackage{caption}
\usepackage{subcaption}

% Formato de página
\usepackage[left=2.5cm,right=2.5cm,top=2.5cm,bottom=2.5cm]{geometry}
\usepackage{fancyhdr}
\usepackage{setspace}

% Colores y cajas
\usepackage[dvipsnames]{xcolor}
\usepackage{tcolorbox}
\usepackage{framed}

% Enlaces e hipervínculos
\usepackage[hidelinks]{hyperref}
\usepackage{url}

% Algoritmos y código
\usepackage{algorithm}
\usepackage{algpseudocode}
\usepackage{listings}

% Bibliografía
\usepackage[numbers,sort&compress]{natbib}

% Otros
\usepackage{lipsum}
\usepackage{enumitem}

% ============================================================================
% CONFIGURACIONES
% ============================================================================

% Configuración de listings para código Python
\lstset{
    language=Python,
    basicstyle=\ttfamily\small,
    keywordstyle=\color{blue},
    commentstyle=\color{gray},
    stringstyle=\color{red},
    showstringspaces=false,
    breaklines=true,
    frame=single,
    numbers=left,
    numberstyle=\tiny\color{gray}
}

% Formato de algoritmos en español
\renewcommand{\algorithmicrequire}{\textbf{Entrada:}}
\renewcommand{\algorithmicensure}{\textbf{Salida:}}
\renewcommand{\algorithmicif}{\textbf{si}}
\renewcommand{\algorithmicthen}{\textbf{entonces}}
\renewcommand{\algorithmicelse}{\textbf{si no}}
\renewcommand{\algorithmicfor}{\textbf{para}}
\renewcommand{\algorithmicwhile}{\textbf{mientras}}
\renewcommand{\algorithmicreturn}{\textbf{retornar}}

% Captions
\captionsetup{font=small,labelfont=bf}

% Interlineado
\onehalfspacing

% Encabezados y pies de página
\pagestyle{fancy}
\fancyhf{}
\fancyhead[LE,RO]{\thepage}
\fancyhead[RE]{\nouppercase{\leftmark}}
\fancyhead[LO]{\nouppercase{\rightmark}}
\renewcommand{\headrulewidth}{0.4pt}

% ============================================================================
% COMANDOS PERSONALIZADOS
% ============================================================================

\newcommand{\VIF}{\mathrm{VIF}}
\newcommand{\TMB}{\mathrm{TMB}}
\newcommand{\FCr}{\mathrm{FC}_r}
\newcommand{\FCw}{\mathrm{FC}_w}
\newcommand{\HRV}{\mathrm{HRV}}
\newcommand{\SDNN}{\mathrm{SDNN}}

% Cajas de definiciones
\newtcolorbox{definicion}{
    colback=blue!5!white,
    colframe=blue!75!black,
    title=Definición
}

% Cajas de notas
\newtcolorbox{nota}{
    colback=yellow!5!white,
    colframe=orange!75!black,
    title=Nota
}

% ============================================================================
% INFORMACIÓN DEL DOCUMENTO
% ============================================================================

\title{
    \textbf{Sistema de Inferencia Difusa para Evaluación del Comportamiento Sedentario mediante Datos de Wearables}\\
    \vspace{0.5cm}
    \large Modelo Mamdani con Validación Cruzada vs. Clustering No Supervisado
}

\author{
    Luis Ángel Martínez\\
    \small Maestría en Ciencias, Semestre 3\\
    \small \texttt{luis.martinez@institution.edu}
}

\date{18 de octubre de 2025}

% ============================================================================
% INICIO DEL DOCUMENTO
% ============================================================================

\begin{document}

% Página de título
\maketitle
\thispagestyle{empty}

% ============================================================================
% RESUMEN / ABSTRACT
% ============================================================================

\begin{abstract}
\noindent
\textbf{Objetivo:} Desarrollar y validar un sistema de inferencia difusa tipo Mamdani para clasificar el sedentarismo semanal a partir de biomarcadores obtenidos de wearables (Apple Watch), contrastando su salida con una verdad operativa derivada de clustering no supervisado K-means.

\textbf{Métodos:} Cohorte de 10 adultos (5M/5H) con seguimiento multianual. Unidad de análisis: 1,337 semanas agregadas. Pipeline de 5 fases: preprocesamiento con imputación jerárquica, creación de variables derivadas (Actividad\_relativa, Superávit\_calórico\_basal), agregación semanal (medianas/IQR), clustering K-means (K=2), y sistema difuso con 4 entradas $\times$ 3 etiquetas y 5 reglas Mamdani. Funciones de membresía triangulares derivadas por percentiles de la cohorte. Validación: búsqueda exhaustiva de umbral $\tau$ que maximiza F1-score.

\textbf{Resultados:} Clustering K=2 óptimo (Silhouette=0.232): Cluster 0 (Bajo Sedentarismo, 402 semanas, 30\%) y Cluster 1 (Alto Sedentarismo, 935 semanas, 70\%). Sistema difuso con umbral $\tau=0.30$: F1-score=0.840, Recall=0.976, Accuracy=0.740, Precision=0.737, MCC=0.294. Matriz de confusión: TN=77, FP=325, FN=22, TP=913. Concordancia por usuario: media 70.0\% (rango 27.7\%--99.3\%).

\textbf{Conclusiones:} Alta sensibilidad (Recall=97.6\%) adecuada para screening poblacional; trade-off aceptado (FP=26\%) con confirmación clínica posterior. Variables normalizadas por antropometría (Superávit\_calórico\_basal) y exposición (Actividad\_relativa) permiten comparabilidad inter-sujeto. Sistema interpretable con reglas auditables converge con estructura data-driven (F1=0.84). Heterogeneidad inter-sujeto sugiere personalización futura de $\tau$ o reglas moduladas por IQR.

\textbf{Palabras clave:} Lógica difusa, sedentarismo, wearables, Apple Watch, clustering, validación cruzada, screening.
\end{abstract}

\newpage
\tableofcontents
\newpage

% ============================================================================
% 1. INTRODUCCIÓN
% ============================================================================

\section{Introducción}

\subsection{Contexto y Motivación}

El sedentarismo es un factor de riesgo cardiovascular y metabólico crítico en poblaciones contemporáneas, asociado con enfermedades cardiovasculares, diabetes tipo 2, obesidad y mortalidad por todas las causas \citep{Troiano2008,TudorLocke2011}. La evaluación objetiva del comportamiento sedentario en vida libre requiere:

\begin{enumerate}[itemsep=0pt]
    \item \textbf{Métricas basadas en wearables:} Superan el sesgo de autoreporte \citep{Troiano2008}.
    \item \textbf{Normalización antropométrica:} 400 kcal gastadas no representan impacto fisiológico equivalente en sujetos con distinto metabolismo basal (TMB).
    \item \textbf{Integración multivariada:} Actividad física + eficiencia autonómica (HRV) + carga cardíaca ($\Delta$ FC).
\end{enumerate}

\subsection{Justificación de Variables Clave}

\subsubsection{Actividad Relativa (Normalizada por Exposición)}

\begin{definicion}
\textbf{Actividad\_relativa} se define como:
\begin{equation}
\text{Actividad\_relativa} = \frac{\text{min\_totales\_en\_movimiento}}{60 \times \text{Total\_hrs\_monitoreadas}}
\end{equation}
\end{definicion}

\textbf{Rationale fisiológico:} Dos usuarios con 30 minutos de movimiento NO son equivalentes si uno tiene 24h monitoreadas (1.25\% del tiempo) y otro solo 8h (6.25\% del tiempo). Esta variable corrige por exposición de monitoreo, permitiendo comparabilidad justa.

\textbf{Evidencia empírica:} Normalizar por tiempo de uso mejora la comparabilidad inter-sujeto en estudios con acelerómetros \citep{Troiano2008,TudorLocke2011}.

\subsubsection{Superávit Calórico Basal (Ajustado por TMB)}

\begin{definicion}
\textbf{Superávit\_calórico\_basal} se define como:
\begin{equation}
\text{Superávit\_calórico\_basal} = \frac{\text{Gasto\_calorico\_activo} \times 100}{\TMB}
\end{equation}

Donde $\TMB$ (Tasa Metabólica Basal) se calcula mediante la ecuación de Mifflin-St Jeor \citep{Mifflin1990}:

\textbf{Hombres:}
\begin{equation}
\TMB = 10 \cdot \text{peso(kg)} + 6.25 \cdot \text{altura(cm)} - 5 \cdot \text{edad} + 5
\end{equation}

\textbf{Mujeres:}
\begin{equation}
\TMB = 10 \cdot \text{peso(kg)} + 6.25 \cdot \text{altura(cm)} - 5 \cdot \text{edad} - 161
\end{equation}
\end{definicion}

\textbf{Ejemplo clínico:} Usuario 9 (124 kg, hombre, $\TMB \approx 2241$ kcal/día): 400 kcal activas = 17.8\% del TMB. Usuario 10 (58 kg, mujer, $\TMB \approx 1304$ kcal/día): 400 kcal activas = 30.7\% del TMB. El impacto fisiológico es distinto y es capturado por el ratio.

\subsubsection{HRV\_SDNN (Variabilidad Cardíaca)}

La variabilidad de la frecuencia cardíaca (SDNN, \textit{Standard Deviation of NN intervals}) es un marcador del tono vagal:
\begin{itemize}
    \item \textbf{HRV alta ($>60$ ms):} Tono vagal saludable, recuperación adecuada.
    \item \textbf{HRV baja ($<40$ ms):} Estrés crónico, desacondicionamiento, riesgo cardiovascular aumentado \citep{Thayer2010,TaskForce1996}.
\end{itemize}

\textbf{Uso en sistema difuso:} HRV baja refuerza la clasificación de sedentarismo alto (marcador indirecto de desacondicionamiento).

\subsubsection{Delta Cardiaco (Respuesta al Esfuerzo)}

\begin{definicion}
\textbf{Delta\_cardiaco} se define como:
\begin{equation}
\Delta_{\text{cardiaco}} = \FCw_{\text{p50}} - \FCr_{\text{p50}}
\end{equation}
donde $\FCw$ es la frecuencia cardíaca al caminar y $\FCr$ la frecuencia cardíaca en reposo.
\end{definicion}

\textbf{Interpretación fisiológica:}
\begin{itemize}
    \item $\Delta$ alto ($>50$ lpm): Respuesta cardíaca apropiada al esfuerzo.
    \item $\Delta$ bajo ($<30$ lpm): Posible bloqueo beta, desacondicionamiento o bajo esfuerzo percibido.
\end{itemize}

\subsection{Por Qué Lógica Difusa}

La Tabla~\ref{tab:fuzzy_vs_clustering} compara clustering K-means (enfoque data-driven) con lógica difusa (enfoque interpretable).

\begin{table}[h]
\centering
\caption{Comparación entre Clustering y Lógica Difusa}
\label{tab:fuzzy_vs_clustering}
\small
\begin{tabularx}{\textwidth}{l X X}
\toprule
\textbf{Aspecto} & \textbf{Clustering (K-Means)} & \textbf{Lógica Difusa} \\
\midrule
Interpretabilidad & Partición dura; centroides abstractos & Reglas lingüísticas auditables \\
Personalización & Reasignación global (retraining) & Ajuste local de MF o pesos \\
Explicabilidad & ``Perteneces al cluster 1'' & ``Score alto por: Act. baja ($\mu=0.9$) $\land$ HRV baja ($\mu=0.7$)'' \\
Validación externa & Requiere recalibración completa & MF por percentiles $\rightarrow$ fácil recalibración \\
\bottomrule
\end{tabularx}
\end{table}

\textbf{Decisión:} Usar clustering como verdad operativa para descubrir estructura data-driven, luego validar sistema fuzzy interpretable contra esa estructura.

\subsection{Objetivos}

\textbf{Objetivo principal:} Desarrollar y validar un sistema de inferencia difusa tipo Mamdani para clasificar el sedentarismo semanal, con alta sensibilidad (adecuado para screening) y reglas interpretables clínicamente.

\textbf{Objetivos específicos:}
\begin{enumerate}
    \item Preprocesar datos de Apple Watch con imputación jerárquica sin leak temporal.
    \item Crear variables derivadas normalizadas por antropometría y exposición.
    \item Descubrir estructura latente mediante clustering no supervisado (K-means).
    \item Diseñar funciones de membresía triangulares por percentiles de la cohorte.
    \item Definir reglas difusas basadas en conocimiento clínico y estructura de clusters.
    \item Validar sistema fuzzy vs. clustering, buscando umbral $\tau$ que maximiza F1-score.
    \item Caracterizar heterogeneidad inter-sujeto y proponer estrategias de personalización.
\end{enumerate}

% ============================================================================
% 2. DATOS Y COHORTE
% ============================================================================

\section{Datos y Cohorte}

\subsection{Características de la Cohorte}

La Tabla~\ref{tab:cohorte} describe las características antropométricas de los 10 participantes.

\begin{table}[h]
\centering
\caption{Características Antropométricas y Seguimiento de la Cohorte}
\label{tab:cohorte}
\small
\begin{tabular}{lcccccccc}
\toprule
\textbf{Usuario} & \textbf{Sexo} & \textbf{Edad} & \textbf{Peso} & \textbf{Estatura} & \textbf{TMB} & \textbf{Días} & \textbf{Semanas} \\
 &  & (años) & (kg) & (cm) & (kcal/día) & monitor. & válidas \\
\midrule
u1 (ale) & M & 34 & 68 & 170 & 1411 & 1048 & 149 \\
u2 (brenda) & M & 37 & 76 & 169 & 1476 & 56 & 7 \\
u3 (christina) & M & 39 & 77 & 164 & 1445 & 1001 & 141 \\
u4 (edson) & H & 25 & 100 & 180 & 2013 & 110 & 14 \\
u5 (esmeralda) & M & 28 & 64 & 160 & 1329 & 104 & 14 \\
u6 (fidel) & H & 34 & 100 & 180 & 1958 & 1967 & 278 \\
u7 (kevin) & H & 32 & 92 & 156 & 1717 & 812 & 114 \\
u8 (legarda) & H & 29 & 92 & 181 & 1893 & 1345 & 191 \\
u9 (lmartinez) & H & 32 & 124 & 185 & 2241 & 2070 & 298 \\
u10 (vane) & M & 28 & 58 & 164 & 1304 & 925 & 131 \\
\bottomrule
\end{tabular}
\end{table}

\textbf{Observaciones:}
\begin{itemize}
    \item Rango de TMB: 1,304--2,241 kcal/día (variabilidad 72\%).
    \item Duración de seguimiento: 56--2,070 días (heterogeneidad temporal alta).
    \item Usuarios u2, u4, u5: $<20$ semanas válidas (contribuyen $\sim$2.6\% del dataset).
\end{itemize}

\subsection{Variables Semanales para Modelado}

La unidad de análisis final es la \textbf{semana} (7 días consecutivos, válida si $\geq 5$ días con uso $\geq 8$h/día). Se calculan 8 features robustos (medianas e IQR) por variable clave:

\begin{itemize}[itemsep=0pt]
    \item \texttt{Actividad\_relativa\_p50}, \texttt{Actividad\_relativa\_iqr}
    \item \texttt{Superavit\_calorico\_basal\_p50}, \texttt{Superavit\_calorico\_basal\_iqr}
    \item \texttt{HRV\_SDNN\_p50}, \texttt{HRV\_SDNN\_iqr}
    \item \texttt{Delta\_cardiaco\_p50}, \texttt{Delta\_cardiaco\_iqr}
\end{itemize}

\textbf{Rationale de usar medianas/IQR:} Robustas a outliers diarios (errores de sensor, eventos únicos). Capturan tendencia central y variabilidad sin asumir normalidad. IQR como proxy de intermitencia conductual (alto IQR = días muy dispares).

% ============================================================================
% 3. PIPELINE METODOLÓGICO
% ============================================================================

\section{Pipeline Metodológico}

La Figura~\ref{fig:pipeline} esquematiza las 5 fases del procesamiento.

\begin{figure}[h]
\centering
\fbox{\parbox{0.95\textwidth}{
\textbf{FASE 1:} Preprocesamiento Diario + Imputación Jerárquica\\
$\downarrow$\\
\textbf{FASE 2:} Creación de Variables Derivadas\\
$\downarrow$\\
\textbf{FASE 3:} Agregación Semanal Robusta (p50, IQR)\\
$\downarrow$\\
\textbf{FASE 4:} Clustering No Supervisado (K=2, verdad operativa)\\
$\downarrow$\\
\textbf{FASE 5:} Sistema Difuso + Validación vs. Clusters
}}
\caption{Pipeline de procesamiento de datos (5 fases)}
\label{fig:pipeline}
\end{figure}

\subsection{Fase 1: Preprocesamiento e Imputación Jerárquica}

\subsubsection{Gates de Clasificación de Días}

\textbf{Gate 1 (Hard No-Wear):}
\begin{equation}
\text{hard\_nowear} = (\text{Total\_hrs\_monitorizadas} < 8) \lor (\text{Hrs\_sin\_registro} > 16)
\end{equation}
\textbf{Acción:} NO imputar. Dejar NaN y marcar fuente = ``sin\_imputar\_hard''.

\textbf{Gate 2 (Soft Low-Activity):}
\begin{equation}
\text{soft\_lowact} = \neg\text{hard\_nowear} \land (8 \leq \text{Total\_hrs} < 12) \land (\text{pasos} < 800)
\end{equation}
\textbf{Acción:} Imputar con baseline fisiológica $\FCw_{\text{imput}} = \FCr + \Delta^*$, donde:
\begin{equation}
\Delta^* = \text{median}\{\FCw_{\text{obs}}[i] - \FCr[i] : i \in \text{días\_observados}\}
\end{equation}

\textbf{Gate 3 (Normal):} Método primario = rolling mediana del pasado (ventana=7 días, soporte$\geq 4$). Si soporte insuficiente, fallback a baseline $\FCr + \Delta^*$.

\begin{nota}
\textbf{Garantía de no-leakage:} Rolling mediana solo usa valores $[t-w, t-1]$ (pasado). Cuantiles acumulados calculados sobre datos $[0, t-1]$ sin incluir $t$.
\end{nota}

\subsection{Fase 2: Variables Derivadas}

Tras imputación jerárquica, se calculan:
\begin{align}
\text{Actividad\_relativa} &= \frac{\text{min\_mov}}{60 \times \text{hrs\_monitor}} \\
\TMB &= f(\text{sexo}, \text{peso}, \text{altura}, \text{edad}) \quad \text{(Mifflin-St Jeor)} \\
\text{Superavit\_cal\_basal} &= \frac{\text{Gasto\_activo} \times 100}{\TMB}
\end{align}

\textbf{Reemplazos:} Eliminar \texttt{min\_totales\_en\_movimiento} y \texttt{Gasto\_calorico\_activo} originales para evitar multicolinealidad (VIF$>10$ si se mantienen ambos).

\subsection{Fase 3: Agregación Semanal}

Para cada variable clave, se calculan \textbf{p50} (mediana) y \textbf{IQR} (rango intercuartílico) por semana. El dataset final semanal contiene 1,385 semanas, de las cuales 1,337 son válidas tras filtrar:
\begin{itemize}
    \item Semanas con $<3$ días monitoreados.
    \item Semanas con $>60\%$ de imputación en FC\_al\_caminar.
\end{itemize}

\subsection{Fase 4: Clustering K-Means}

\textbf{Configuración:}
\begin{itemize}
    \item Algoritmo: K-Means (sklearn, init=`k-means++', random\_state=42).
    \item Features: 8 variables semanales escaladas con RobustScaler.
    \item K-sweep: $K \in \{2, 3, 4, 5, 6\}$.
    \item Métricas: Silhouette, Davies-Bouldin, estabilidad ARI (20 bootstraps).
\end{itemize}

\textbf{Selección de K:} $K=2$ (mejor Silhouette=0.232). La Tabla~\ref{tab:k_sweep} resume el K-sweep.

\begin{table}[h]
\centering
\caption{Resultados del K-sweep para Clustering}
\label{tab:k_sweep}
\small
\begin{tabular}{ccccl}
\toprule
\textbf{K} & \textbf{Silhouette} & \textbf{Davies-Bouldin} & \textbf{ARI} & \textbf{Tamaños} \\
\midrule
2 & \textbf{0.232} & 2.058 & 0.565 & \{0: 402, 1: 935\} \\
3 & 0.195 & 1.721 & 0.654 & \{0: 685, 1: 235, 2: 417\} \\
4 & 0.192 & 1.422 & 0.735 & \{0: 238, 1: 662, 2: 435, 3: 2\} \\
5 & 0.148 & 1.444 & 0.446 & \{0: 213, 1: 375, 2: 544, 3: 1, 4: 204\} \\
6 & 0.159 & 1.430 & 0.777 & \{0: 204, 1: 456, 2: 200, 3: 337, 4: 139, 5: 1\} \\
\bottomrule
\end{tabular}
\end{table}

\textbf{Perfiles clínicos de K=2:}
\begin{itemize}
    \item \textbf{Cluster 0 (Bajo Sedentarismo):} 402 semanas (30\%). Actividad\_rel=0.160, Superávit=45.4\%, HRV=47.7 ms.
    \item \textbf{Cluster 1 (Alto Sedentarismo):} 935 semanas (70\%). Actividad\_rel=0.116, Superávit=25.4\%, HRV=49.5 ms.
\end{itemize}

\subsection{Fase 5: Sistema de Inferencia Difusa}

\subsubsection{Variables Lingüísticas de Entrada}

Se definen 4 variables de entrada, cada una con 3 etiquetas (Baja/Media/Alta), usando funciones de membresía \textbf{triangulares} derivadas por percentiles de la cohorte.

\textbf{Ejemplo (Actividad\_relativa\_p50):}
\begin{align}
\mu_{\text{Baja}}(x) &: \text{Triangular}(p_{10}, p_{25}, p_{40}) \\
\mu_{\text{Media}}(x) &: \text{Triangular}(p_{35}, p_{50}, p_{65}) \\
\mu_{\text{Alta}}(x) &: \text{Triangular}(p_{60}, p_{75}, p_{90})
\end{align}

Las Figuras~\ref{fig:mf_actividad}--\ref{fig:mf_delta} muestran las funciones de membresía para las 4 variables.

\begin{figure}[h]
\centering
\includegraphics[width=0.8\textwidth]{../analisis_u/fuzzy/plots/MF_Actividad_relativa_p50.png}
\caption{Funciones de membresía para Actividad\_relativa\_p50}
\label{fig:mf_actividad}
\end{figure}

\begin{figure}[h]
\centering
\includegraphics[width=0.8\textwidth]{../analisis_u/fuzzy/plots/MF_Superavit_calorico_basal_p50.png}
\caption{Funciones de membresía para Superávit\_calórico\_basal\_p50}
\label{fig:mf_superavit}
\end{figure}

\begin{figure}[h]
\centering
\includegraphics[width=0.8\textwidth]{../analisis_u/fuzzy/plots/MF_HRV_SDNN_p50.png}
\caption{Funciones de membresía para HRV\_SDNN\_p50}
\label{fig:mf_hrv}
\end{figure}

\begin{figure}[h]
\centering
\includegraphics[width=0.8\textwidth]{../analisis_u/fuzzy/plots/MF_Delta_cardiaco_p50.png}
\caption{Funciones de membresía para Delta\_cardiaco\_p50}
\label{fig:mf_delta}
\end{figure}

\subsubsection{Base de Reglas Difusas}

Se definen 5 reglas tipo Mamdani (operador AND = mínimo, OR = máximo):

\begin{enumerate}
    \item \textbf{R1:} SI Actividad es \textit{Baja} Y Superávit es \textit{Bajo} ENTONCES Sedentarismo es \textit{Alto}.
    \item \textbf{R2:} SI Actividad es \textit{Alta} Y Superávit es \textit{Alto} ENTONCES Sedentarismo es \textit{Bajo}.
    \item \textbf{R3:} SI HRV es \textit{Baja} Y Delta es \textit{Alta} ENTONCES Sedentarismo es \textit{Alto}.
    \item \textbf{R4:} SI Actividad es \textit{Media} Y HRV es \textit{Media} ENTONCES Sedentarismo es \textit{Medio}.
    \item \textbf{R5:} SI Actividad es \textit{Baja} Y Superávit es \textit{Medio} ENTONCES Sedentarismo es \textit{Medio-Alto} (peso 0.7).
\end{enumerate}

\subsubsection{Defuzzificación}

Salida: Sedentarismo\_score $\in [0, 1]$. Método: centroide (centro de gravedad del área agregada). Binarización con umbral $\tau=0.30$ (maximiza F1 vs. clusters).

\begin{equation}
\hat{y}_i = \begin{cases} 1 \, (\text{Alto Sed}) & \text{si Score}_i \geq \tau \\ 0 \, (\text{Bajo Sed}) & \text{si Score}_i < \tau \end{cases}
\end{equation}

% ============================================================================
% 4. RESULTADOS
% ============================================================================

\section{Resultados}

\subsection{Distribución del Score Difuso}

La Figura~\ref{fig:score_hist} muestra el histograma del Sedentarismo\_score.

\begin{figure}[h]
\centering
\includegraphics[width=0.75\textwidth]{../analisis_u/fuzzy/plots/sedentarismo_score_histogram.png}
\caption{Distribución del Sedentarismo\_score (N=1,385 semanas)}
\label{fig:score_hist}
\end{figure}

\textbf{Estadísticas:} Media=0.571 $\pm$ 0.235, Rango=[0.000, 1.000] $\rightarrow$ distribución no degenerada.

\subsection{Validación vs. Clustering}

\subsubsection{Búsqueda de Umbral Óptimo}

Grid search $\tau \in [0.05, 0.95]$ con paso 0.05; métrica objetivo = max(F1). \textbf{Resultado:} $\tau=0.30$.

\subsubsection{Métricas Globales}

La Tabla~\ref{tab:metricas_globales} resume las métricas de validación.

\begin{table}[h]
\centering
\caption{Métricas de Validación Fuzzy vs. Clustering (N=1,337)}
\label{tab:metricas_globales}
\begin{tabular}{lc}
\toprule
\textbf{Métrica} & \textbf{Valor} \\
\midrule
Umbral óptimo ($\tau$) & 0.300 \\
Accuracy & 0.740 \\
F1-Score & \textbf{0.840} \\
Precision & 0.737 \\
Recall (Sensibilidad) & \textbf{0.976} \\
MCC & 0.294 \\
\bottomrule
\end{tabular}
\end{table}

\subsubsection{Matriz de Confusión}

La Tabla~\ref{tab:confusion} y Figura~\ref{fig:confusion} muestran la matriz de confusión.

\begin{table}[h]
\centering
\caption{Matriz de Confusión (Cluster vs. Fuzzy)}
\label{tab:confusion}
\begin{tabular}{l|cc|c}
\toprule
 & \textbf{Fuzzy: Bajo} & \textbf{Fuzzy: Alto} & \textbf{Total} \\
\midrule
\textbf{Cluster: Bajo (0)} & TN = 77 & FP = 325 & 402 \\
\textbf{Cluster: Alto (1)} & FN = 22 & TP = 913 & 935 \\
\midrule
\textbf{Total} & 99 & 1,238 & 1,337 \\
\bottomrule
\end{tabular}
\end{table}

\begin{figure}[h]
\centering
\includegraphics[width=0.7\textwidth]{../analisis_u/fuzzy/plots/confusion_matrix.png}
\caption{Matriz de confusión visual}
\label{fig:confusion}
\end{figure}

\textbf{Interpretación:}
\begin{itemize}
    \item \textbf{TN (77):} Semanas correctamente identificadas como Bajo Sedentarismo.
    \item \textbf{TP (913):} Semanas correctamente identificadas como Alto Sedentarismo.
    \item \textbf{FP (325):} Cluster dice ``Bajo'', fuzzy dice ``Alto'' $\rightarrow$ sobrediagnóstico (política conservadora).
    \item \textbf{FN (22):} Cluster dice ``Alto'', fuzzy dice ``Bajo'' $\rightarrow$ subdiagnóstico (bajo, deseable minimizar).
\end{itemize}

\subsubsection{Curva Precision-Recall}

La Figura~\ref{fig:pr_curve} muestra la curva PR para distintos valores de $\tau$.

\begin{figure}[h]
\centering
\includegraphics[width=0.75\textwidth]{../analisis_u/fuzzy/plots/pr_curve.png}
\caption{Curva Precision-Recall. Punto óptimo: $\tau=0.30$ (F1=0.84)}
\label{fig:pr_curve}
\end{figure}

\subsection{Análisis por Usuario}

La Tabla~\ref{tab:per_user} resume la concordancia por usuario.

\begin{table}[h]
\centering
\caption{Concordancia Fuzzy vs. Clustering por Usuario}
\label{tab:per_user}
\small
\begin{tabular}{lcccccc}
\toprule
\textbf{Usuario} & \textbf{N} & \textbf{Acc} & \textbf{F1} & \textbf{Rec} & \textbf{TP} & \textbf{FP} \\
\midrule
u1 (ale) & 149 & 0.993 & 0.997 & 1.000 & 148 & 1 \\
u2 (brenda) & 7 & 0.429 & 0.600 & 0.500 & 3 & 1 \\
u3 (christina) & 141 & 0.277 & 0.215 & 0.875 & 14 & 100 \\
u4 (edson) & 14 & 0.714 & 0.833 & 1.000 & 10 & 4 \\
u5 (esmeralda) & 14 & 0.714 & 0.818 & 1.000 & 9 & 4 \\
u6 (fidel) & 278 & 0.817 & 0.898 & 0.982 & 224 & 47 \\
u7 (kevin) & 114 & 0.947 & 0.973 & 1.000 & 108 & 6 \\
u8 (legarda) & 191 & 0.440 & 0.462 & 0.868 & 46 & 100 \\
u9 (lmartinez) & 298 & 0.856 & 0.919 & 0.976 & 245 & 37 \\
u10 (vane) & 131 & 0.809 & 0.895 & 1.000 & 106 & 25 \\
\bottomrule
\end{tabular}
\end{table}

\textbf{Concordancia media:} 70.0\% (rango: 27.7\%--99.3\%).

\textbf{Usuarios con alta concordancia ($>$90\%):} u1 (ale), u7 (kevin) $\rightarrow$ Patrones estables.

\textbf{Usuarios con baja concordancia ($<$50\%):} u3 (christina), u8 (legarda) $\rightarrow$ Alta variabilidad intra-semanal.

% ============================================================================
% 5. DISCUSIÓN
% ============================================================================

\section{Discusión}

\subsection{Validez Clínica del Sistema Fuzzy}

\textbf{1. Alta sensibilidad (Recall=97.6\%):} Minimiza falsos negativos (FN=22/935=2.4\%). Solo $\sim$2\% de semanas verdaderamente sedentarias pasan desapercibidas. \textbf{Implicación:} Apto para cribado poblacional; casos positivos fuzzy pueden confirmarse con evaluación clínica adicional.

\textbf{2. Trade-off Precision vs. Recall:} Precision=73.7\% implica que $\sim$26\% de clasificaciones ``Alto Sedentarismo'' son FP (cluster los marcó como ``Bajo''). \textbf{Justificación:} En salud pública, preferible alertar de más (con confirmación posterior) que pasar por alto casos de riesgo.

\textbf{3. Roles fisiológicos confirmados:} Actividad\_relativa y Superávit\_calórico\_basal son principales discriminadores (pesos altos en reglas R1, R2, R5). HRV\_SDNN complementario; HRV baja + actividad baja refuerza clasificación de Alto Sedentarismo (R3).

\subsection{Heterogeneidad Inter-Sujeto}

\textbf{Observación:} Concordancia usuario-específica varía 27.7\%--99.3\%.

\textbf{Hipótesis explicativas:}
\begin{enumerate}
    \item \textbf{Variabilidad intra-semanal (IQR alto):} Usuarios con alta intermitencia (días muy activos vs. muy sedentarios) $\rightarrow$ clustering agrupa por ``promedios'', fuzzy captura extremos.
    \item \textbf{Tamaño de muestra desigual:} u2, u4, u5 tienen $<20$ semanas $\rightarrow$ estadística poco robusta.
    \item \textbf{Perfiles extremos:} u1 (ale): 99.3\% concordancia $\rightarrow$ patrón muy estable. u8 (legarda): 44.0\% concordancia $\rightarrow$ perfil atípico (actividad moderada-baja pero HRV alta).
\end{enumerate}

\textbf{Estrategias de mitigación:}
\begin{itemize}
    \item $\tau$ personalizado por usuario (requiere $\geq 30$ semanas).
    \item Reglas moduladas por IQR: atenuar peso de R1 en alta intermitencia.
    \item Validación en subpoblaciones (por sexo, rango de TMB).
\end{itemize}

\subsection{Comparación con Clustering}

El sistema fuzzy ofrece \textbf{transparencia y auditabilidad} necesarias para aplicaciones clínicas, sin perder concordancia significativa con estructura data-driven del clustering (F1=0.84). Ventaja sobre K-means: reglas directamente revisables por clínico, ajuste local de MF sin reentrenar todo el modelo.

% ============================================================================
% 6. FORTALEZAS Y LIMITACIONES
% ============================================================================

\section{Fortalezas Metodológicas}

\begin{enumerate}
    \item \textbf{Convergencia supervisado--no supervisado:} Sistema fuzzy interpretable converge con estructura data-driven (F1=0.84) $\rightarrow$ validación cruzada robusta.
    \item \textbf{MF por percentiles (data-driven):} Anclaje a distribución observada $\rightarrow$ robustez ante outliers. Fácil recalibración en nueva cohorte (recalcular percentiles, mantener estructura triangular).
    \item \textbf{Trazabilidad completa:} Auditorías de imputación (\texttt{FC\_walk\_imputacion\_V3.csv} con fuentes y flags), logs por paso, reproducibilidad garantizada (semillas fijas, config en YAML).
    \item \textbf{Garantías de no-leakage:} Rolling mediana solo hacia atrás, cuantiles acumulados sin incluir $t$, auditoría temporal explícita en logs.
\end{enumerate}

\section{Limitaciones y Mitigación}

\begin{enumerate}
    \item \textbf{Falsos positivos (FP=325):} Requiere zona gris (scores 0.40--0.60) con etiqueta ``Indeterminado'' + evaluación clínica.
    \item \textbf{Cohorte pequeña (N=10):} Validación externa necesaria antes de despliegue clínico (cohorte $\geq 20$ usuarios, $\geq 1000$ semanas).
    \item \textbf{Silhouette moderado (K=2: 0.232):} Refleja continuo fisiológico (no frontera ``dura''). K=2 preferido por interpretabilidad clínica clara sobre complejidad algorítmica.
    \item \textbf{Escalado global:} Recalibración anual o por cohorte para evitar arrastre por valores extremos históricos.
\end{enumerate}

% ============================================================================
% 7. CONCLUSIONES
% ============================================================================

\section{Conclusiones}

\begin{enumerate}
    \item \textbf{Sistema fuzzy validado:} Convergencia robusta con clustering K=2 (F1=0.84, Recall=97.6\%). Reglas interpretables capturan estructura real del sedentarismo en la cohorte.
    
    \item \textbf{Política conservadora efectiva:} Alta sensibilidad minimiza falsos negativos, adecuado para screening poblacional con confirmación clínica posterior. Trade-off FP aceptado (26\%).
    
    \item \textbf{Variables fisiológicamente relevantes:} Actividad\_relativa y Superávit\_calórico\_basal (principales discriminadores) + HRV\_SDNN y Delta\_cardiaco (complementarios) $\rightarrow$ integración multivariada robusta.
    
    \item \textbf{Heterogeneidad manejable:} Concordancia usuario-específica 27.7\%--99.3\% $\rightarrow$ personalización futura necesaria ($\tau$ ajustable, reglas moduladas por IQR).
    
    \item \textbf{Trazabilidad y reproducibilidad:} Pipeline completo documentado, auditorías de imputación, fácil recalibración en nueva cohorte (recalcular percentiles de MF).
\end{enumerate}

% ============================================================================
% 8. PRÓXIMOS PASOS
% ============================================================================

\section{Próximos Pasos}

\subsection{Corto Plazo}
\begin{enumerate}
    \item Personalización de $\tau$ por usuario o subpoblaciones.
    \item Reglas moduladas por IQR para capturar intermitencia.
    \item Análisis de sensibilidad de MF (variar percentiles $\pm 5\%$, medir impacto en F1).
\end{enumerate}

\subsection{Mediano-Largo Plazo}
\begin{enumerate}
    \item Validación externa en nueva cohorte ($\geq 20$ usuarios, $\geq 1000$ semanas).
    \item Integración de nuevas variables: sueño (duración, eficiencia), estrés percibido.
    \item Modelado temporal avanzado: ARIMA/LSTM para predecir ``próxima semana será Alto Sedentarismo''.
    \item Implementación de dashboard clínico (FastAPI + React + Plotly).
    \item Publicación científica en revista de salud digital (\textit{JMIR mHealth}, \textit{Digital Health}).
\end{enumerate}

% ============================================================================
% BIBLIOGRAFÍA
% ============================================================================

\bibliographystyle{plain}
\bibliography{referencias}

% ============================================================================
% APÉNDICES
% ============================================================================

\appendix

\section{Pseudocódigo del Sistema Fuzzy}

\begin{algorithm}
\caption{Sistema de Inferencia Difusa para Sedentarismo}
\begin{algorithmic}[1]
\Require $X_{n \times 8}$ (features semanales), $\text{MF\_config}$ (funciones de membresía), $\text{rules}$ (5 reglas), $\tau=0.30$
\Ensure $\hat{y} \in \{0,1\}^n$ (clasificación binaria)
\For{$i = 1$ \textbf{to} $n$}
    \State $\mathbf{x}_i \gets X[i, :]$ \Comment{Fila $i$ de features}
    \State // \textbf{Fuzzificación}
    \For{cada variable $v$ en \{Actividad, Superávit, HRV, Delta\}}
        \For{cada etiqueta $\ell$ en \{Baja, Media, Alta\}}
            \State $\mu_{v,\ell}(\mathbf{x}_i) \gets \text{TriangularMF}(\mathbf{x}_i[v], \text{MF\_config}[v][\ell])$
        \EndFor
    \EndFor
    \State // \textbf{Activación de Reglas}
    \For{cada regla $r \in \text{rules}$}
        \State $w_r \gets \min\{\mu_{v,\ell}(\mathbf{x}_i) : (v,\ell) \in \text{antecedentes}(r)\}$ \Comment{AND = mínimo}
    \EndFor
    \State // \textbf{Agregación}
    \State $s_{\text{Bajo}} \gets \max\{w_r : \text{conseq}(r) = \text{Bajo}\}$ \Comment{OR = máximo}
    \State $s_{\text{Medio}} \gets \max\{w_r : \text{conseq}(r) = \text{Medio}\}$
    \State $s_{\text{Alto}} \gets \max\{w_r : \text{conseq}(r) = \text{Alto}\}$
    \State // \textbf{Defuzzificación (centroide)}
    \State $\text{Score}_i \gets \frac{s_{\text{Bajo}} \cdot 0.2 + s_{\text{Medio}} \cdot 0.5 + s_{\text{Alto}} \cdot 0.8}{s_{\text{Bajo}} + s_{\text{Medio}} + s_{\text{Alto}}}$
    \State // \textbf{Binarización}
    \State $\hat{y}_i \gets \mathbb{1}[\text{Score}_i \geq \tau]$
\EndFor
\State \Return $\hat{y}$
\end{algorithmic}
\end{algorithm}

\section{Reproducibilidad}

Todos los artefactos están disponibles en:
\begin{itemize}
    \item \textbf{Configuración fuzzy:} \texttt{fuzzy\_config/fuzzy\_membership\_config.yaml}
    \item \textbf{Salidas fuzzy:} \texttt{analisis\_u/fuzzy/fuzzy\_output.csv}
    \item \textbf{Evaluación:} \texttt{analisis\_u/fuzzy/09\_eval\_fuzzy\_vs\_cluster.txt}
    \item \textbf{Semanal consolidado:} \texttt{weekly\_consolidado.csv}
    \item \textbf{Clustering:} \texttt{analisis\_u/clustering/cluster\_assignments.csv}
\end{itemize}

Semillas fijas: \texttt{random\_state=42} en todos los modelos. Logs de auditoría por paso.

\end{document}



