%!TEX program = pdflatex
% ============================================================================
% PRESENTACIÓN BEAMER: Sistema de Inferencia Difusa para Sedentarismo
% ============================================================================

\documentclass[aspectratio=169]{beamer}

% ============================================================================
% TEMA Y COLORES
% ============================================================================

\usetheme{Madrid}
\usecolortheme{whale}

% Colores personalizados
\definecolor{AzulOscuro}{RGB}{0,51,102}
\definecolor{AzulMedio}{RGB}{51,102,153}
\definecolor{VerdeAcento}{RGB}{0,176,80}

\setbeamercolor{title}{fg=white,bg=AzulOscuro}
\setbeamercolor{frametitle}{fg=white,bg=AzulMedio}
\setbeamercolor{structure}{fg=AzulMedio}
\setbeamercolor{item}{fg=AzulMedio}

% ============================================================================
% PAQUETES
% ============================================================================

\usepackage[utf8]{inputenc}
\usepackage[spanish]{babel}
\usepackage{graphicx}
\usepackage{booktabs}
\usepackage{amsmath}
\usepackage{multirow}
\usepackage{array}
\usepackage{tikz}

% ============================================================================
% INFORMACIÓN DEL DOCUMENTO
% ============================================================================

\title[Sistema Difuso Sedentarismo]{Sistema de Inferencia Difusa para Evaluación del Comportamiento Sedentario mediante Datos de Wearables}

\subtitle{Modelo Mamdani con Validación vs. Clustering No Supervisado}

\author[L.A. Martínez]{Luis Ángel Martínez}

\institute[Universidad]{
    Maestría en Ciencias, Semestre 3\\
    Universidad [Nombre]\\
    \texttt{luis.martinez@institution.edu}
}

\date{18 de octubre de 2025}

% Numeración de slides
\setbeamertemplate{footline}[frame number]

% ============================================================================
% INICIO DEL DOCUMENTO
% ============================================================================

\begin{document}

% ============================================================================
% SLIDE 1: TÍTULO
% ============================================================================

\begin{frame}
\titlepage
\end{frame}

% ============================================================================
% SLIDE 2: TABLA DE CONTENIDOS
% ============================================================================

\begin{frame}{Contenido}
\tableofcontents
\end{frame}

% ============================================================================
% SECCIÓN 1: INTRODUCCIÓN
% ============================================================================

\section{Introducción y Objetivos}

\begin{frame}{Contexto y Motivación}

\begin{block}{Desafío Clínico}
El sedentarismo es un factor de riesgo cardiovascular crítico
\begin{itemize}
    \item Evaluación objetiva mediante wearables (Apple Watch)
    \item 400 kcal gastadas $\neq$ impacto fisiológico equivalente
    \item Necesidad de normalización antropométrica
\end{itemize}
\end{block}

\vspace{0.3cm}

\begin{columns}[T]
\column{0.5\textwidth}
\textbf{Ventajas de Lógica Difusa:}
\begin{itemize}
    \item Reglas interpretables
    \item Personalización local
    \item Explicabilidad clínica
\end{itemize}

\column{0.5\textwidth}
\textbf{Clustering como Verdad:}
\begin{itemize}
    \item Estructura data-driven
    \item Validación objetiva
    \item Sin sesgo de diseño
\end{itemize}
\end{columns}

\end{frame}

% ----------------------------------------------------------------------------

\begin{frame}{Objetivos}

\begin{block}{Objetivo Principal}
Desarrollar y validar un \textbf{sistema de inferencia difusa tipo Mamdani} para clasificar el sedentarismo semanal, con alta sensibilidad y reglas clínicamente interpretables
\end{block}

\vspace{0.5cm}

\textbf{Objetivos Específicos:}
\begin{enumerate}
    \item Preprocesar datos con imputación jerárquica sin leak temporal
    \item Crear variables derivadas normalizadas (Actividad\_relativa, Superávit\_calórico\_basal)
    \item Descubrir estructura latente con clustering K-means (K=2)
    \item Diseñar funciones de membresía triangulares por percentiles
    \item Definir reglas difusas clínicamente interpretables
    \item Validar sistema fuzzy vs. clustering (búsqueda de $\tau$ óptimo)
\end{enumerate}

\end{frame}

% ============================================================================
% SECCIÓN 2: DATOS Y METODOLOGÍA
% ============================================================================

\section{Datos y Metodología}

\begin{frame}{Datos y Cohorte}

\begin{columns}[T]
\column{0.4\textwidth}
\textbf{Características:}
\begin{itemize}
    \item Cohorte: 10 adultos (5M/5H)
    \item Seguimiento: Multianual
    \item Unidad: 1,337 semanas válidas
    \item Fuente: Apple Watch
\end{itemize}

\vspace{0.3cm}

\textbf{Rango de TMB:}
\begin{itemize}
    \item Min: 1,304 kcal/d (u10)
    \item Max: 2,241 kcal/d (u9)
    \item Variabilidad: 72\%
\end{itemize}

\column{0.6\textwidth}
\begin{table}
\centering
\small
\begin{tabular}{lccccc}
\toprule
\textbf{Usuario} & \textbf{S} & \textbf{Edad} & \textbf{Peso} & \textbf{TMB} & \textbf{N} \\
\midrule
u1 (ale) & M & 34 & 68 & 1411 & 149 \\
u6 (fidel) & H & 34 & 100 & 1958 & 278 \\
u9 (lmartinez) & H & 32 & 124 & 2241 & 298 \\
u10 (vane) & M & 28 & 58 & 1304 & 131 \\
\midrule
... & ... & ... & ... & ... & ... \\
\bottomrule
\end{tabular}
\end{table}
\end{columns}

\end{frame}

% ----------------------------------------------------------------------------

\begin{frame}{Variables Derivadas Clave (1/2)}

\begin{block}{1. Actividad\_relativa (normalizada por exposición)}
\begin{equation*}
\text{Actividad\_relativa} = \frac{\text{min\_movimiento}}{60 \times \text{hrs\_monitoreadas}}
\end{equation*}
\textbf{Rationale:} Corrige por tiempo de uso del reloj. 30 min en 8h $\neq$ 30 min en 24h.
\end{block}

\vspace{0.3cm}

\begin{block}{2. Superávit\_calórico\_basal (ajustado por TMB)}
\begin{equation*}
\text{Superávit} = \frac{\text{Gasto\_activo} \times 100}{\text{TMB}}
\end{equation*}

Donde TMB (Mifflin-St Jeor):
\begin{align*}
\text{Hombres:} & \quad \text{TMB} = 10 \cdot \text{peso} + 6.25 \cdot \text{altura} - 5 \cdot \text{edad} + 5 \\
\text{Mujeres:} & \quad \text{TMB} = 10 \cdot \text{peso} + 6.25 \cdot \text{altura} - 5 \cdot \text{edad} - 161
\end{align*}

\textbf{Rationale:} 400 kcal en TMB=2241 (17.8\%) $\neq$ 400 kcal en TMB=1304 (30.7\%)
\end{block}

\end{frame}

% ----------------------------------------------------------------------------

\begin{frame}{Variables Derivadas Clave (2/2)}

\begin{columns}[T]
\column{0.5\textwidth}
\begin{block}{3. HRV\_SDNN (variabilidad cardíaca)}
Marcador del tono vagal:
\begin{itemize}
    \item Alta HRV ($>60$ ms): Tono vagal saludable
    \item Baja HRV ($<40$ ms): Desacondicionamiento, estrés crónico
\end{itemize}

\vspace{0.2cm}

\textbf{Uso en fuzzy:} HRV baja refuerza clasificación de Alto Sedentarismo
\end{block}

\column{0.5\textwidth}
\begin{block}{4. Delta\_cardiaco (respuesta al esfuerzo)}
\begin{equation*}
\Delta_{\text{cardiaco}} = \text{FC}_{\text{caminata}} - \text{FC}_{\text{reposo}}
\end{equation*}

\vspace{0.2cm}

Interpretación:
\begin{itemize}
    \item $\Delta$ alto ($>50$ lpm): Respuesta apropiada
    \item $\Delta$ bajo ($<30$ lpm): Posible desacondicionamiento
\end{itemize}
\end{block}
\end{columns}

\end{frame}

% ----------------------------------------------------------------------------

\begin{frame}{Pipeline Metodológico (5 Fases)}

\begin{center}
\begin{tikzpicture}[node distance=1.5cm, auto]
\tikzstyle{block} = [rectangle, draw, fill=blue!20, text width=6cm, text centered, rounded corners, minimum height=1cm]
\tikzstyle{arrow} = [thick,->,>=stealth]

\node [block] (fase1) {\textbf{FASE 1:} Preprocesamiento + Imputación Jerárquica};
\node [block, below of=fase1] (fase2) {\textbf{FASE 2:} Variables Derivadas};
\node [block, below of=fase2] (fase3) {\textbf{FASE 3:} Agregación Semanal (p50, IQR)};
\node [block, below of=fase3] (fase4) {\textbf{FASE 4:} Clustering K=2 (Verdad Operativa)};
\node [block, below of=fase4] (fase5) {\textbf{FASE 5:} Sistema Difuso + Validación};

\draw [arrow] (fase1) -- (fase2);
\draw [arrow] (fase2) -- (fase3);
\draw [arrow] (fase3) -- (fase4);
\draw [arrow] (fase4) -- (fase5);
\end{tikzpicture}
\end{center}

\end{frame}

% ============================================================================
% SECCIÓN 3: CLUSTERING
% ============================================================================

\section{Clustering No Supervisado}

\begin{frame}{Clustering K-Means (Verdad Operativa)}

\begin{columns}[T]
\column{0.45\textwidth}
\textbf{Configuración:}
\begin{itemize}
    \item Algoritmo: K-Means
    \item Features: 8 variables semanales (p50 + IQR)
    \item K-sweep: K $\in \{2,3,4,5,6\}$
    \item Escalado: RobustScaler
\end{itemize}

\vspace{0.3cm}

\textbf{Selección de K=2:}
\begin{itemize}
    \item Silhouette máximo: 0.232
    \item Interpretación clínica clara
    \item Estabilidad ARI: 0.565
\end{itemize}

\column{0.55\textwidth}
\begin{table}
\centering
\small
\begin{tabular}{cccc}
\toprule
\textbf{K} & \textbf{Silhouette} & \textbf{D-B} & \textbf{Tamaños} \\
\midrule
\textbf{2} & \textbf{0.232} & 2.058 & \{0:402, 1:935\} \\
3 & 0.195 & 1.721 & \{0:685, 1:235, 2:417\} \\
4 & 0.192 & 1.422 & 4 clusters \\
5 & 0.148 & 1.444 & 5 clusters \\
\bottomrule
\end{tabular}
\end{table}

\vspace{0.3cm}

\textbf{Perfiles de K=2:}
\begin{itemize}
    \item \textbf{Cluster 0:} 402 semanas (30\%), Act\_rel=0.160, Sup=45.4\%
    \item \textbf{Cluster 1:} 935 semanas (70\%), Act\_rel=0.116, Sup=25.4\%
\end{itemize}
\end{columns}

\end{frame}

% ============================================================================
% SECCIÓN 4: SISTEMA DIFUSO
% ============================================================================

\section{Sistema de Inferencia Difusa}

\begin{frame}{Funciones de Membresía: Actividad\_relativa}

\begin{center}
\includegraphics[width=0.8\textwidth]{../analisis_u/fuzzy/plots/MF_Actividad_relativa_p50.png}
\end{center}

\begin{itemize}
    \item \textbf{Diseño data-driven:} Percentiles p10-p25-p40, p35-p50-p65, p60-p75-p90
    \item \textbf{Interpretación:} Mayor actividad relativa = MENOR sedentarismo
\end{itemize}

\end{frame}

% ----------------------------------------------------------------------------

\begin{frame}{Funciones de Membresía: Superávit\_calórico\_basal}

\begin{center}
\includegraphics[width=0.8\textwidth]{../analisis_u/fuzzy/plots/MF_Superavit_calorico_basal_p50.png}
\end{center}

\begin{itemize}
    \item \textbf{Rango:} 12.2\% - 76.7\% del TMB
    \item \textbf{Interpretación:} Mayor superávit = MENOR sedentarismo
\end{itemize}

\end{frame}

% ----------------------------------------------------------------------------

\begin{frame}{Funciones de Membresía: HRV\_SDNN y Delta\_cardiaco}

\begin{columns}[T]
\column{0.5\textwidth}
\begin{center}
\includegraphics[width=\textwidth]{../analisis_u/fuzzy/plots/MF_HRV_SDNN_p50.png}

\textbf{HRV\_SDNN}

Menor HRV = MAYOR sedentarismo (desacondicionamiento)
\end{center}

\column{0.5\textwidth}
\begin{center}
\includegraphics[width=\textwidth]{../analisis_u/fuzzy/plots/MF_Delta_cardiaco_p50.png}

\textbf{Delta\_cardiaco}

Delta alto = buena respuesta cardíaca al esfuerzo
\end{center}
\end{columns}

\end{frame}

% ----------------------------------------------------------------------------

\begin{frame}{Sistema de Inferencia Difusa (5 Reglas)}

\textbf{Entradas (4):} Actividad\_relativa, Superávit\_calórico, HRV\_SDNN, Delta\_cardiaco

\vspace{0.5cm}

\textbf{Reglas Mamdani (operador AND = mínimo):}

\begin{enumerate}
    \item \textbf{R1:} SI Actividad es \textit{Baja} Y Superávit es \textit{Bajo} $\Rightarrow$ Sedentarismo \textit{Alto}
    \item \textbf{R2:} SI Actividad es \textit{Alta} Y Superávit es \textit{Alto} $\Rightarrow$ Sedentarismo \textit{Bajo}
    \item \textbf{R3:} SI HRV es \textit{Baja} Y Delta es \textit{Alta} $\Rightarrow$ Sedentarismo \textit{Alto}
    \item \textbf{R4:} SI Actividad es \textit{Media} Y HRV es \textit{Media} $\Rightarrow$ Sedentarismo \textit{Medio}
    \item \textbf{R5:} SI Actividad es \textit{Baja} Y Superávit es \textit{Medio} $\Rightarrow$ Sedentarismo \textit{Medio-Alto}
\end{enumerate}

\vspace{0.3cm}

\textbf{Defuzzificación:} Centroide $\rightarrow$ Score $\in [0, 1]$

\textbf{Binarización:} $\tau = 0.30$ (maximiza F1 vs. clusters)

\end{frame}

% ============================================================================
% SECCIÓN 5: RESULTADOS
% ============================================================================

\section{Resultados y Validación}

\begin{frame}{Métricas de Validación}

\begin{center}
\Huge
\textbf{F1-Score: 0.840}

\vspace{0.3cm}

\textbf{Recall: 97.6\%}

\vspace{0.3cm}

\large
Accuracy: 74.0\% \quad Precision: 73.7\% \quad MCC: 0.294
\end{center}

\vspace{0.5cm}

\begin{columns}[T]
\column{0.5\textwidth}
\textbf{Matriz de Confusión:}
\begin{itemize}
    \item TN = 77 (Bajo correcto)
    \item FP = 325 (Sobreclas. conservador)
    \item FN = 22 (Subclasificación baja)
    \item TP = 913 (Alto correcto)
\end{itemize}

\column{0.5\textwidth}
\textbf{Interpretación:}
\begin{itemize}
    \item \textcolor{VerdeAcento}{\textbf{Alta sensibilidad (97.6\%):}} Solo 22/935 semanas pasan desapercibidas
    \item \textcolor{orange}{\textbf{Trade-off FP (26\%):}} Política conservadora para screening
\end{itemize}
\end{columns}

\end{frame}

% ----------------------------------------------------------------------------

\begin{frame}{Matriz de Confusión (Visual)}

\begin{center}
\includegraphics[width=0.7\textwidth]{../analisis_u/fuzzy/plots/confusion_matrix.png}
\end{center}

\textbf{Concordancia:} 990/1,337 semanas (74.0\%)

\end{frame}

% ----------------------------------------------------------------------------

\begin{frame}{Curva Precision-Recall y Distribución de Scores}

\begin{columns}[T]
\column{0.5\textwidth}
\begin{center}
\includegraphics[width=\textwidth]{../analisis_u/fuzzy/plots/pr_curve.png}

\textbf{Curva PR}

Punto óptimo: $\tau=0.30$ (F1=0.84)
\end{center}

\column{0.5\textwidth}
\begin{center}
\includegraphics[width=\textwidth]{../analisis_u/fuzzy/plots/score_distribution_by_cluster.png}

\textbf{Distribución por Cluster}

Cluster 0: Score=0.454$\pm$0.249

Cluster 1: Score=0.621$\pm$0.212
\end{center}
\end{columns}

\end{frame}

% ----------------------------------------------------------------------------

\begin{frame}{Concordancia por Usuario (Heterogeneidad)}

\begin{columns}[T]
\column{0.45\textwidth}
\textbf{Concordancia media:} 70.0\%

\textbf{Rango:} 27.7\%--99.3\%

\vspace{0.5cm}

\textbf{Alta concordancia ($>90\%$):}
\begin{itemize}
    \item u1 (ale): 99.3\%
    \item u7 (kevin): 94.7\%
    \item $\rightarrow$ Patrones estables
\end{itemize}

\vspace{0.3cm}

\textbf{Baja concordancia ($<50\%$):}
\begin{itemize}
    \item u3 (christina): 27.7\%
    \item u8 (legarda): 44.0\%
    \item $\rightarrow$ Alta variabilidad intra-semanal
\end{itemize}

\column{0.55\textwidth}
\begin{table}
\centering
\small
\begin{tabular}{lccc}
\toprule
\textbf{Usuario} & \textbf{Conc.} & \textbf{F1} & \textbf{Recall} \\
\midrule
u1 (ale) & \textbf{99.3\%} & 0.997 & 1.000 \\
u7 (kevin) & \textbf{94.7\%} & 0.973 & 1.000 \\
u6 (fidel) & 81.7\% & 0.898 & 0.982 \\
u10 (vane) & 80.9\% & 0.895 & 1.000 \\
u3 (christina) & 27.7\% & 0.215 & 0.875 \\
u8 (legarda) & 44.0\% & 0.462 & 0.868 \\
\bottomrule
\end{tabular}
\end{table}
\end{columns}

\end{frame}

% ============================================================================
% SECCIÓN 6: CONCLUSIONES
% ============================================================================

\section{Conclusiones y Próximos Pasos}

\begin{frame}{Conclusiones Principales}

\begin{enumerate}
    \item \textbf{Sistema fuzzy validado:}
    \begin{itemize}
        \item Convergencia robusta con clustering K=2 (F1=0.84, Recall=97.6\%)
        \item Reglas interpretables capturan estructura real del sedentarismo
    \end{itemize}
    
    \item \textbf{Política conservadora efectiva:}
    \begin{itemize}
        \item Alta sensibilidad minimiza falsos negativos $\rightarrow$ Screening poblacional
        \item Trade-off FP aceptado (26\%) con confirmación clínica posterior
    \end{itemize}
    
    \item \textbf{Variables fisiológicamente relevantes:}
    \begin{itemize}
        \item Actividad\_relativa y Superávit\_calórico\_basal (principales)
        \item HRV\_SDNN y Delta\_cardiaco (complementarios)
    \end{itemize}
    
    \item \textbf{Heterogeneidad manejable:}
    \begin{itemize}
        \item Concordancia usuario-específica 27.7\%--99.3\%
        \item Personalización futura necesaria ($\tau$ ajustable, reglas por IQR)
    \end{itemize}
    
    \item \textbf{Trazabilidad y reproducibilidad completa}
\end{enumerate}

\end{frame}

% ----------------------------------------------------------------------------

\begin{frame}{Próximos Pasos}

\begin{block}{Corto Plazo}
\begin{itemize}
    \item Personalización de $\tau$ por usuario o subpoblaciones (sexo, rango de TMB)
    \item Reglas moduladas por IQR para capturar intermitencia conductual
    \item Análisis de sensibilidad de MF (variar percentiles $\pm 5\%$, medir impacto en F1)
\end{itemize}
\end{block}

\begin{block}{Mediano Plazo}
\begin{itemize}
    \item Validación externa en nueva cohorte ($\geq 20$ usuarios, $\geq 1,000$ semanas)
    \item Integración de nuevas variables: Sueño (duración, eficiencia), estrés percibido
    \item Zona gris (scores 0.40--0.60) $\rightarrow$ Etiqueta `Indeterminado' + evaluación adicional
\end{itemize}
\end{block}

\begin{block}{Largo Plazo}
\begin{itemize}
    \item Modelado temporal avanzado: ARIMA/LSTM para predicción de tendencias
    \item Dashboard clínico interactivo (FastAPI + React + Plotly)
    \item Publicación científica, despliegue operativo en salud ocupacional
\end{itemize}
\end{block}

\end{frame}

% ============================================================================
% SLIDE FINAL
% ============================================================================

\begin{frame}[plain]
\begin{center}
\Huge
¡Gracias por su atención!

\vspace{1cm}

\Large
Preguntas y Discusión

\vspace{1cm}

\normalsize
\textbf{Luis Ángel Martínez}\\
Maestría en Ciencias, Semestre 3\\
\texttt{luis.martinez@institution.edu}

\vspace{0.5cm}

\textit{Sistema de Inferencia Difusa para Evaluación de Sedentarismo}\\
\textit{Validado con F1=0.84, Recall=97.6\%}
\end{center}
\end{frame}

\end{document}





